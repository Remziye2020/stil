\addchap{Preface}

\section*{What readers can expect from this book}


\section*{How did I get here?}


\section*{New types of sources and references}

Books and published papers are undoubtedly the primary source of knowledge about statistics and philosphy, even after the year 2015.
However, I have acquired a great deal of my knowledge of statistics not just by reading books and research papers, but from respected online sources.
I benefitted a lot from blogs written by renowned statisticians, from the Stanford Encyclopedia of Philosophy (\url{https://plato.stanford.edu/}), and from  \textit{CrossValidated} (\url{https://stats.stackexchange.com/}).
Especially the blogs and contributions of the following people have been very inspiring and helpful, regardless of how often I reference them in this book (in alphabetical order):

\begin{itemize}
  \item BenBolker (\url{https://stats.stackexchange.com/users/2126/ben-bolker})
  \item Andrew Gelman (\url{http://andrewgelman.com/})
  \item Debora G.\ Mayo (\url{https://errorstatistics.com/})
  \item Richard Morey (\url{http://bayesfactor.blogspot.de/})
  \item Stephen Senn (\url{https://errorstatistics.com/tag/stephen-senn/})
  \item as well as many contributors on R-bloggers (\url{https://www.r-bloggers.com/})
\end{itemize}


\section*{A word on typesetting and technology}

This book was created using \XeLaTeX, which is in many ways superior to the old \LaTeX.
Mixing \XeLaTeX and R code was made easy and elegant by knitr (\url{https://yihui.name/knitr/}), and I recommend anyone to use these three pieces of software to typeset their documents whenever they use statistics.
Since I run an RStudio Server installation at \url{https://webcorpora.org}, I was able to work on this book in a perfectly platform- and machine-independent way.
GitHub (\url{https://github.com/rsling/smil}) provided version control, and the available sources on GitHub can be used by anyone to see how it is done.
