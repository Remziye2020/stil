\addchap{Abbreviations and symbols}

\section*{Abbreviations}

\begin{longtable}{p{0.15\textwidth}p{0.9\textwidth}}
  ANOVA & analysis of variance \\
  CDF   & cumulative distribution function \\
  CLT   & central limit theorem \\
  cp.   & ceteris paribus (all other things being equal) \\
  iid.  & independent and identically distributed \\
  LM    & linear model \\
  LMM   & linear mixed model \\
  GLM   & linear mixed model \\
  GLMM  & generalised linear mixed model \\
  PDF   & probability density function \\
  VCOV  & variance-covariance matrix \\
\end{longtable}


\section*{Symbols}

Symbols are overloaded ad-hoc to denote either a (possibly indexed) value such as $s_x=1$ (for ``the population mean of variable $x$ is $1$'') or a function such as $s(x)=1$ where applicable.

\begin{longtable}{p{0.1\textwidth}p{0.1\textwidth}p{0.8\textwidth}}
	\multicolumn{3}{c}{\textbf{Mathematische Symbole}} \\
  $x\thicksim \mathbf{D}$   &              & Zufallsvariable $x$ \textit{folgt} der Zufallsverteilung $\mathbf{D}$ \\
  $\bar{x}$        &  $mean(x)$    & arithmetisches Stichprobenmittel von $x$\\
  $\tilde{x}$      &  $med(x)$     & Stichprobenmedian von $x$\\
  $\hat{x}$        &               & Vorhergesagter Wert für $x$\\
                   &               & \\
	\multicolumn{3}{c}{\textbf{Symbole in Buchstabenform}} \\

  $\alpha$          &              & Alphaniveau (Neyman-Pearson, NHST) \\
  $\alpha_i$        &              & Intercept $i$ \\
  $\beta$           &              & Betaniveau (Neyman-Pearson) \\
  $\beta_i$         &              & Koeffizient $i$ der ersten Ebene (hierarchische Modelle) \\
  $df$              &              & Freiheitsgrade (\textit{degrees of freedom}) \\
  $e$               &              & Euler-Konstante \\
  $\epsilon$        &              & Fehler/Residuum auf Beobachtungsebene (Modelle) \\
  $f$               &              & Frequenz \\
  $F$               &              & F-Quotien (siehe ANOVA)\\
  $\gamma_i$        &              & Koeffizient $i$ der zweiten Ebene (hierarchische Modelle)  \\
                    & $cov(x, y)$  & Kovarianz von $x$ und $y$ \\
  $H$               &              & Kruskal-Wallis-Statistik \\
  $H_0$             &              & Nullhypothese (Fisher) \\
  $H_A$             &              & Alternativhypothese (Neyman-Pearson) \\
  $M_M$             &              & Haupthypothese  (Neyman-Pearson) \\
  $IQR$             &              & Interquartislabstand \\
  $\mathcal{L}$     &              & Likelihood \\
  $\mu$             &              & Populationsmittel \\
  $\mu_i$           &              & Mittel des modellierten Effekts $i$ (Modelle) \\
  $n$               &              & Stichprobengröße \\
  $N$               &              & Populationsgröße \\
  $O$               &              & Chance (\textit{odds}) \\
  $p$               &              & Anteilswert (\textit{proportion}) \\
  $P_i$             &              & $i$-tes Perzentil \\
  $Pr$              &              & Wahrscheinlichkeit (\textit{probability}) \\
  $\varphi$         &              & Dispersionsparameter \\
  $Q_i$             &              & $i$-tes Quartil \\
  $r$               &              & Stichproben-Kovariations-Koeffizient \\
  $r^2$             &              & Bestimmtheitsmaß (\textit{coefficient of determination}) \\
  $R^2$             &              & multifaktorielles Bestimmtheitsmaß \\
  $\rho$            &              & Grundgesamtheits-Kovariations-Koeffizient \\
  $s_x$             & $sd(x)$      & Standardabweichung der Stichprobe $x$ (\textit{standard deviation}) \\
  $s^2_x$           & $var(x)$     & Varianz der Stichprobe $x$ \\
  $SE_{n_x,\pi_x}$      & $se(n_x, \pi_x)$ & Standardfehler (\textit{standard error}) für $n_x$ und $\pi_x$ \\
  $SQ_{x,y}$        & $sq(x, y)$   & Quadratsumme (\textit{sum of squares}) für Stichproben $x$ und $y$ \\
  $SP_{x,y}$        & $sp(x, y)$   & Porduktsumme (\textit{sum of products}) für Stichproben $x$ und $y$ \\
  $\sigma$          & $sd(X)$      & Standardabweichung der Population \\
  $\sigma^2$        &              & Varianz der Population \\
  $sig$             &              & Signifikanzniveau (Fisher) \\
  $U$               &              & Mann-Whitney-Statistik \\
  $VCOV_m$          & $vcov(m)$    & Varianz-Kovarianz-Matrix von $m$ \\
  $\chi^2$          &              & Chi-Quadrat-Statistik \\
\end{longtable}

\vspace{\baselineskip}
\noindent Zuvallsverteilungen werden hier durch fettgedruckte Buchtsbaben gekennzeichnet, nicht durch den manchmal üblichen Skript-Font.
Die kumulative Verteilung wird mit Strich gegeben, also \textbf{Norm\Prime}.
\vspace{\baselineskip}

\begin{longtable}{p{0.1\textwidth}p{0.1\textwidth}p{0.8\textwidth}}
  $\mathbf{Bern}$   &  $\mathcal{B}$   & Bernoulliverteilung \\
  $\mathbf{Exp}$    &  $\mathcal{E}$   & Exponentialverteilung \\
  $\mathbf{F}$      &  $\mathcal{F}$   & $F$-Verteilung \\
  $\mathbf{Norm}$   &  $\mathcal{N}$   & Normal-\slash Gauss-Verteilung \\
  $\mathbf{t}$      &  $\mathcal{T}$   & t-Verteilung \\
  $\mathbf{Unif}$   &  $\mathcal{U}$   & uniforme Verteilung \\
  $\mathbf{Chisq}$  &  $\mathcal{X^2}$ & $\chi^2$-Verteilung \\
\end{longtable}
